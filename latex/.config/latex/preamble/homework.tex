%%% PACKAGES %%%
\usepackage{amsmath}
\usepackage{amsthm}
\usepackage{amssymb}
\usepackage{amsfonts}
\usepackage{color}
\usepackage{microtype}
\usepackage{lmodern}
\usepackage[mathscr]{eucal}
\usepackage{bm}
\usepackage{enumitem}
\usepackage{graphicx}
\usepackage[portuguese]{babel}
\usepackage{hyperref}
\usepackage[affil-it]{authblk} % para afiliação
\usepackage{makeidx} % para índice remissivo
\usepackage{import} % para importar arquivos
%\usepackage{makecell}
%\usepackage{colortbl}
%\usepackage[dvipsnames]{xcolor}


%%% MACROS %%%
\def\R{\mathbb{R}}
\def\C{\mathscr{C}}
\def\Z{\mathbb{Z}}
\def\N{\mathbb{N}}
\def\Q{\mathbb{Q}}
\def\V{\mathcal{V}}
\def\U{\mathscr{U}}
\def\B{\mathcal{B}}
\def\P{\mathbb{P}}
\def\Pcal{\mathcal{P}}
\def\H{\mathbb{H}}
\def\X{\mathscr{X}}
\def\inv{^{-1}}
\def\spam{\operatorname{spam}}
\def\ans{\textnormal{\textbf{Solução: }}}
\def\Id{\operatorname{Id}}
\def\limn{\lim_{n\rightarrow \infty}}
\def\e{\quad\text{e}\quad}
\def\opt{{\operatorname{opt}}}
\def\sign{{\operatorname{sign}}}
\def\pto{\stackrel{p}{\to}}
\def\dto{\stackrel{d}{\to}}
\def\iid{\stackrel{\text{iid}}{\sim}}
\def\d{\operatorname{d}}
\def\std{\operatorname{std}}
\newcommand{\normal}[1]{\mathcal{N}\left(#1\right)}
\newcommand{\ntexttt}[1]{\texttt{\textup{#1}}}
\newcommand{\abs}[1]{\left\vert#1\right\vert}
\newcommand{\inner}[1]{\left\langle #1\right\rangle}
\newcommand{\re}[1]{\operatorname{Re}(#1)}
\newcommand{\im}[1]{\operatorname{Im}(#1)}
\newcommand{\sgn}[1]{\operatorname{sgn}(#1)}
\newcommand{\norm}[1]{\left\lVert#1\right\rVert}
\newcommand{\normp}[2][p]{\parent{\int |#2|^#1}^{1/#1}}
\newcommand{\del}[2]{\dfrac{\partial #1}{\partial #2}}
\newcommand{\deldel}[3]{\dfrac{\partial^2 #1}{\partial #3\partial #2}}
\newcommand{\indicadora}[1]{\mathbf{1}_{\left[ #1 \right]}}
\newcommand{\E}[1]{\mathbb{E}\left[#1\right]}
\newcommand{\var}[1]{\mathbb{V}\left[#1\right]}
\newcommand{\cov}[1]{\operatorname{Cov}\left[#1\right]}
\newcommand{\EP}[2]{{\mathbb{E}}_{#1}\left[#2\right]}
\newcommand{\seq}[1]{(#1_n)_{n\in\N}}
\newcommand{\under}[2]{\underset{#2}{{\underbrace{#1}}}} % para escrever texto embaixo de um brace.
\DeclareMathOperator*{\argmax}{arg\,max}
\DeclareMathOperator*{\argmin}{arg\,min}




% %%% SETTINGS %%%
% \setlist[enumerate,1]{label=\textnormal{(\alph*)}}
% \setlist[enumerate,2]{label=\textnormal{(\roman*)}}
% %\renewcommand{\thesubsection}{\arabic{subsection}}
% %\renewcommand{\thesection}{}
% \renewcommand{\contentsname}{Exercícios}
% \renewcommand{\solutiontitle}{\noindent\textnormal{\textbf{Solução:}}\nopagebreak\par\noindent}
% \renewcommand\partlabel{\textnormal{(\alph{partno})}}
% \graphicspath{ {./images/} }
% \usepackage[margin=1.0in]{geometry}
% \pagestyle{plain}
% \renewcommand{\clearpage}{}

\makeatletter
\def\@maketitle{%
  \newpage
  \null
  \vskip 2em%
  \begin{center}%
  \let \footnote \thanks
    {\Large\bfseries \@title \par}%
    \vskip 1.5em%
    {\normalsize
      \lineskip .5em%
      \begin{tabular}[t]{c}%
        \@author
      \end{tabular}\par}%
    \vskip 1em%
    {\normalsize \@date}%
  \end{center}%
  \par
  \vskip 1.5em}
\makeatother


%%% LISTINGS %%%
\usepackage{listings}
\lstset{ 
  language=Python,
  basicstyle=\normalfont\ttfamily\small,
  numbers=none,
  frame=lines,
  framexleftmargin=0.5em,
  xleftmargin=0.5em,
  framexrightmargin=0.5em,
  rulecolor=\color{black},
  belowcaptionskip=1\baselineskip,
  breaklines=true,
  showstringspaces=false,
  belowskip=1em,
  aboveskip=1em,
  linewidth=\linewidth,
  float=htb
  captionpos=b
  morekeywords={as}, % Add 'as' as an additional keyword
  keywordstyle=\color{blue}\textbf, % Set the color for keywords
   literate={á}{{\'a}}1
           {â}{{\^a}}1
           {ã}{{\~a}}1
           {à}{{\`a}}1
           {é}{{\'e}}1
           {ê}{{\^e}}1
           {í}{{\'i}}1
           {ó}{{\'o}}1
           {ô}{{\^o}}1
           {õ}{{\~o}}1
           {ú}{{\'u}}1
           {ü}{{\"u}}1
           {ç}{{\c{c}}}1
}



% % Questions in italic
% \let\oldquestions\questions
% \let\endoldquestions\endquestions
% \renewenvironment{questions}
%   {\oldquestions\itshape}
%   {\endoldquestions}
  
% % Solutions not in italic
% \let\oldsolution\solution
% \let\endoldsolution\endsolution
% \renewenvironment{solution}{%
%   \begin{oldsolution}%
%   \normalshape% Make the text inside italic
% }{%
%   \end{oldsolution}%
% }


% bibliography
\usepackage[numbers,sort,compress]{natbib}

%%% Algorithms
\usepackage{algorithm}
\usepackage[noend]{algpseudocode}

\usepackage{listings}

% Default fixed font does not support bold face
\DeclareFixedFont{\ttb}{T1}{txtt}{bx}{n}{10} % for bold
\DeclareFixedFont{\ttm}{T1}{txtt}{m}{n}{10}  % for normal

% Custom colors
\definecolor{deepblue}{rgb}{0,0,0.5}
\definecolor{deepred}{rgb}{0.6,0,0}
\definecolor{deepgreen}{rgb}{0,0.5,0}

% Python style for highlighting
\newcommand\pythonstyle{\lstset{
language=Python,
basicstyle=\ttm,
otherkeywords={self},             % Add keywords here
keywordstyle=\ttb\color{deepblue},
emph={MyClass,__init__},          % Custom highlighting
emphstyle=\ttb\color{deepred},    % Custom highlighting style
stringstyle=\color{deepgreen},
commentstyle=\ttfamily\small\color{deepred},
frame=tb,                         % Any extra options here
showstringspaces=false,            % 
extendedchars=true,
literate={á}{{\'a}}1 {ã}{{\~a}}1 {õ}{{\~o}}1 {é}{{\'e}}1 {í}{{\'i}}1 {ç}{{\c{c}}}1 {Ç}{{\c{C}}}1 {ú}{{\'u}}1 {*}{{*}}1,
}}

% Python environment
\lstnewenvironment{python}[1][]
{
\pythonstyle
\lstset{#1}
}
{}

% Python for external files
\newcommand\pythonexternal[2][]{{
\pythonstyle
\lstinputlisting[#1]{#2}}}

% Python for inline
\newcommand\pythoninline[1]{{\pythonstyle\lstinline!#1!}}


% R style for highlighting
\newcommand\Rstyle{\lstset{
language=R,
basicstyle=\ttm,
otherkeywords={self},             % Add keywords here
keywordstyle=\ttb\color{deepblue},
emph={MyClass,__init__},          % Custom highlighting
emphstyle=\ttb\color{deepred},    % Custom highlighting style
stringstyle=\color{deepgreen},
frame=tb,                         % Any extra options here
showstringspaces=false            % 
}}

% Python environment
\lstnewenvironment{Rcode}[1][]
{
\Rstyle
\lstset{#1}
}
{}

% R for external files (always .R)
\newcommand\Rexternal[2][]{{
  \Rstyle
  \lstinputlisting[#1]{#2} % always add .R extension
}}

% remove o label das questões
% \qformat{} 


%%% COMMANDS %%%
% To add a section to the table of contents without numbering it
\newcommand*\silentsection[1]{%
\addtocounter{section}{1}
\addcontentsline{toc}{section}{\protect\numberline{\thesection}#1}
\sectionmark{#1}
}

% To add a subsection to the table of contents without numbering it
\newcommand*\silentsubsection[1]{%
  \addtocounter{subsection}{1}
  \addcontentsline{toc}{subsection}{\protect\numberline{\thesubsection}#1}
  \subsectionmark{#1}
}


% Save the original definition of \question
% \let\oldquestion\question

\newcommand*{\ssquestion}{%
  \par                % <<-- ensure vertical mode
  \question
  \phantomsection
  \silentsubsection{Question \thequestion}%
}
