%%% PACKAGES %%%
\usepackage{amsmath}
\usepackage{amsthm}
\usepackage{amssymb}
\usepackage{amsfonts}
\usepackage{mathtools}
\usepackage{color}
\usepackage{microtype}
\usepackage{lmodern}
\usepackage[mathscr]{eucal}
\usepackage{bm}
\usepackage{enumitem}
\usepackage{graphicx}
\usepackage{titlesec}
\usepackage[a4paper,margin=1in]{geometry}

\usepackage{marginnote}  % margin notes
\usepackage{stackengine} % for \stackon
\usepackage{scalerel}    % for \scaleto


%%% MACROS %%%
\def\spam{\operatorname{spam}}
\def\Id{\operatorname{Id}}
\def\sign{{\operatorname{sign}}}
\newcommand{\abs}[1]{\left\vert#1\right\vert}
\newcommand{\inner}[1]{\left\langle #1\right\rangle}
\newcommand{\re}[1]{\operatorname{Re}(#1)}
\newcommand{\im}[1]{\operatorname{Im}(#1)}
\newcommand{\sgn}[1]{\operatorname{sgn}(#1)}
\newcommand{\norm}[1]{\left\lVert#1\right\rVert}
\newcommand{\normp}[2][p]{\parent{\int |#2|^#1}^{1/#1}}
\newcommand{\del}[2]{\dfrac{\partial #1}{\partial #2}}
\newcommand{\deldel}[3]{\dfrac{\partial^2 #1}{\partial #3\partial #2}}
\newcommand{\indicator}[1]{\mathbf{1}_{(#1)}}
\newcommand{\E}[1]{\mathbb{E}\left[#1\right]}
\newcommand{\var}[1]{\mathbb{V}\left[#1\right]}
\newcommand{\cov}[1]{\text{Cov}\left[#1\right]}
%\newcommand{\EP}[2]{\underset{#1}{\mathbb{E}}\left[#2\right]}
\newcommand{\EP}[2]{\mathbb{E}_{#1}\left[#2\right]}
\newcommand{\under}[2]{\underset{#2}{{\underbrace{#1}}}} % para escrever texto embaixo de um brace.
\DeclareMathOperator*{\argmax}{arg\,max}
\DeclareMathOperator*{\argmin}{arg\,min}



%%% SETTINGS %%%
% \usepackage{fullpage}  % full page style
\graphicspath{{figures/}}  %  add figures to graphics path
\graphicspath{{../figures/}}  %  add upper figures folder to graphics path


\usepackage[colorlinks = true,
            linkcolor = blue,
            urlcolor  = blue,
            citecolor = blue,
            anchorcolor = blue]{hyperref}

%%% COMMANDS %%%

%%% CODING %%%
\usepackage{listings}
\newcommand\pythonstyle{\lstset{ 
  language=Python,
  basicstyle=\normalfont\ttfamily\small,
  numbers=none,
  frame=lines,
  framexleftmargin=0.5em,
  xleftmargin=0.5em,
  keywordstyle=\ttb\color{deepblue},
  emphstyle=\ttb\color{deepred},    % Custom highlighting style
  stringstyle=\color{deepgreen},
  commentstyle=\ttfamily\small\color{deepred},
  framexrightmargin=0.5em,
  rulecolor=\color{black},
  belowcaptionskip=1\baselineskip,
  breaklines=true,
  showstringspaces=false,
  belowskip=1em,
  aboveskip=1em,
  linewidth=\linewidth,
  captionpos=b
  morekeywords={as}, % Add 'as' as an additional keyword
  keywordstyle=\color{blue}\textbf, % Set the color for keywords
   literate={á}{{\'a}}1 {â}{{\^a}}1 {ã}{{\~a}}1 {à}{{\`a}}1 {é}{{\'e}}1 {ê}{{\^e}}1 {í}{{\'i}}1 {ó}{{\'o}}1 {ô}{{\^o}}1 {õ}{{\~o}}1 {ú}{{\'u}}1 {ü}{{\"u}}1 {ç}{{\c{c}}}1
}}

% Custom colors
\definecolor{deepblue}{rgb}{0,0,0.5}
\definecolor{deepred}{rgb}{0.6,0,0}
\definecolor{deepgreen}{rgb}{0,0.5,0}

% Python environment
\lstnewenvironment{python}[1][]
{
\pythonstyle
\lstset{#1}
}
{}

% Python for external files
\newcommand\pythonexternal[2][]{{
\pythonstyle
\lstinputlisting[#1]{#2}}}

% Python for inline
\newcommand\pythoninline[1]{{\pythonstyle\lstinline!#1!}}

%%% PACKAGES %%%
\usepackage{tcolorbox}
\tcbuselibrary{theorems,skins,breakable, hooks}

%%% COLORS %%%
\definecolor{greentheorembg}{HTML}{E8F5E9}
\definecolor{greentheoremborder}{HTML}{2E7D32}
\definecolor{myb}{HTML}{1E88E5} % for exercises
\definecolor{myp}{RGB}{197, 92, 212}

%%% THEOREM ENVIRONMENTS %%%

\tcbset{
  plainstyle/.style={
    theorem style=plain,
    coltitle=black,
    colback=white,
    colframe=white,
    fonttitle=\bfseries,
    left=-2mm,
    description font=\normalfont,
  }
}

\tcbset{
  plaingreen/.style={
    enhanced,
    breakable,
    colback=greentheorembg,
    colframe=greentheoremborder,
    coltitle=black,
    fonttitle=\bfseries,
    boxrule=0.6pt,
    arc=6pt,
    left=8pt,right=8pt,top=6pt,bottom=6pt,
    theorem style=plain,
    grow to left by=3mm,
    grow to right by=3mm,
  }
}

\newtcbtheorem[number within=section]{theorem}{Theorem}{plaingreen}{thm}
\newtcbtheorem[number within=section]{proposition}{Proposition}{plaingreen}{prop}
\newtcbtheorem[number within=section]{lemma}{Lemma}{plainstyle}{lem}
\newtcbtheorem[number within=section]{corollary}{Corollary}{plainstyle}{cor}
\newtcbtheorem[number within=section]{problem}{Problem}{plainstyle}{prob}
\newtcbtheorem[number within=section]{question}{Question}{plainstyle}{que}
\newtcbtheorem[number within=section]{fact}{Fact}{plainstyle}{fact}
\newtcbtheorem[number within=section]{example}{Example}{plainstyle}{eg}
\newtcbtheorem[number within=section]{definition}{Definition}{plainstyle}{def}
\newtcbtheorem[number within=section]{remark}{Remark}{plainstyle}{rmk}
\newtcbtheorem[number within=section]{solution}{Solution}{plainstyle}{sol}

%%% EXERCISE ENVIRONMENT %%%
\makeatletter
\newtcbtheorem[number from=section]{exercise}{Exercise}{%
  enhanced,
  breakable,
  colback=white,
  colframe=myb!80!black,
  attach boxed title to top left={yshift*=-\tcboxedtitleheight},
  fonttitle=\bfseries,
  title={#2},
  boxed title size=title,
  boxed title style={%
    sharp corners,
    rounded corners=northwest,
    colback=tcbcolframe,
    boxrule=0pt,
  },
  underlay boxed title={%
    \path[fill=tcbcolframe] (title.south west)--(title.south east)
    to[out=0, in=180] ([xshift=5mm]title.east)--
    (title.center-|frame.east)
    [rounded corners=\kvtcb@arc] |-
    (frame.north) -| cycle;
  },
}{def}

% Corollary
\newtcbtheorem[number within=chapter]{pretty}{Pretty}
{%
	enhanced
	,breakable
	,colback = myp!10
	,frame hidden
	,boxrule = 0sp
	,borderline west = {2pt}{0pt}{myp!85!black}
	,sharp corners
	,detach title
	,before upper = \tcbtitle\par\smallskip
	,coltitle = myp!85!black
	,fonttitle = \bfseries\sffamily
	,description font = \mdseries
	,separator sign none
	,segmentation style={solid, myp!85!black}
}
{th}


% Define the danger sign
\newcommand\dangersign[1][1.5ex]{%
  \renewcommand\stacktype{L}%
  \scaleto{\stackon[1pt]{\color{red!80!black}$\triangle$}{\tiny\bfseries !}}{#1}%
}

% Discreet warning: symbol + optional explanatory text in margin
\newcommand{\warning}[1]{%
  \marginnote{\dangersign[3ex]\ \textcolor{red}{\textsf{\small #1}}}%
}

\usepackage{xcolor, tikz}

\newcommand{\classseparator}[1]{%
  \par\vspace{1.2em}\noindent%
  \begin{tikzpicture}
    \draw[gray!60, line width=0.4pt] (0,0) -- (\linewidth,0);
    \node[
      fill=white,
      anchor=center,
      inner sep=2pt,
      text=gray!70,
      font=\sffamily\bfseries\fontsize{9}{11}\selectfont
    ] at (\linewidth/2,0) {#1};
  \end{tikzpicture}%
  \par\vspace{1.2em}
}
