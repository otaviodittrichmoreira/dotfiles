%%% PACKAGES %%%
\usepackage{amsmath}
\usepackage{amsthm}
\usepackage{amssymb}
\usepackage{amsfonts}
\usepackage{mathtools}
\usepackage{color}
\usepackage{microtype}
\usepackage{lmodern}
\usepackage[mathscr]{eucal}
\usepackage{bm}
\usepackage{enumitem}
\usepackage{graphicx}
\usepackage{titlesec}



%%% MACROS %%%
\def\spam{\operatorname{spam}}
\def\Id{\operatorname{Id}}
\def\sign{{\operatorname{sign}}}
\newcommand{\abs}[1]{\left\vert#1\right\vert}
\newcommand{\inner}[1]{\left\langle #1\right\rangle}
\newcommand{\re}[1]{\operatorname{Re}(#1)}
\newcommand{\im}[1]{\operatorname{Im}(#1)}
\newcommand{\sgn}[1]{\operatorname{sgn}(#1)}
\newcommand{\norm}[1]{\left\lVert#1\right\rVert}
\newcommand{\normp}[2][p]{\parent{\int |#2|^#1}^{1/#1}}
\newcommand{\del}[2]{\dfrac{\partial #1}{\partial #2}}
\newcommand{\deldel}[3]{\dfrac{\partial^2 #1}{\partial #3\partial #2}}
\newcommand{\indicator}[1]{\mathbf{1}_{(#1)}}
\newcommand{\E}[1]{\mathbb{E}\left[#1\right]}
\newcommand{\var}[1]{\mathbb{V}\left[#1\right]}
\newcommand{\cov}[1]{\text{Cov}\left[#1\right]}
%\newcommand{\EP}[2]{\underset{#1}{\mathbb{E}}\left[#2\right]}
\newcommand{\EP}[2]{\mathbb{E}_{#1}\left[#2\right]}
\newcommand{\under}[2]{\underset{#2}{{\underbrace{#1}}}} % para escrever texto embaixo de um brace.
\DeclareMathOperator*{\argmax}{arg\,max}
\DeclareMathOperator*{\argmin}{arg\,min}



%%% SETTINGS %%%
% \usepackage{fullpage}  % full page style
\graphicspath{{figures/}}  %  add figures to graphics path
\graphicspath{{../figures/}}  %  add upper figures folder to graphics path


\usepackage[colorlinks = true,
            linkcolor = blue,
            urlcolor  = blue,
            citecolor = blue,
            anchorcolor = blue]{hyperref}

%%% COMMANDS %%%

%%% CODING %%%
\usepackage{listings}
\newcommand\pythonstyle{\lstset{ 
  language=Python,
  basicstyle=\normalfont\ttfamily\small,
  numbers=none,
  frame=lines,
  framexleftmargin=0.5em,
  xleftmargin=0.5em,
  keywordstyle=\ttb\color{deepblue},
  emphstyle=\ttb\color{deepred},    % Custom highlighting style
  stringstyle=\color{deepgreen},
  commentstyle=\ttfamily\small\color{deepred},
  framexrightmargin=0.5em,
  rulecolor=\color{black},
  belowcaptionskip=1\baselineskip,
  breaklines=true,
  showstringspaces=false,
  belowskip=1em,
  aboveskip=1em,
  linewidth=\linewidth,
  captionpos=b
  morekeywords={as}, % Add 'as' as an additional keyword
  keywordstyle=\color{blue}\textbf, % Set the color for keywords
   literate={á}{{\'a}}1 {â}{{\^a}}1 {ã}{{\~a}}1 {à}{{\`a}}1 {é}{{\'e}}1 {ê}{{\^e}}1 {í}{{\'i}}1 {ó}{{\'o}}1 {ô}{{\^o}}1 {õ}{{\~o}}1 {ú}{{\'u}}1 {ü}{{\"u}}1 {ç}{{\c{c}}}1
}}

% Custom colors
\definecolor{deepblue}{rgb}{0,0,0.5}
\definecolor{deepred}{rgb}{0.6,0,0}
\definecolor{deepgreen}{rgb}{0,0.5,0}

% Python environment
\lstnewenvironment{python}[1][]
{
\pythonstyle
\lstset{#1}
}
{}

% Python for external files
\newcommand\pythonexternal[2][]{{
\pythonstyle
\lstinputlisting[#1]{#2}}}

% Python for inline
\newcommand\pythoninline[1]{{\pythonstyle\lstinline!#1!}}


%%%% THEOREM ENVIRONMENTS %%%%%
\usepackage{comment}  % used to hide solutions and comments

%%% Plain style: italic text; use same counter
\theoremstyle{plain}  
	\newtheorem{theorem}{Theorem}
	\newtheorem{proposition}[theorem]{Proposition}
	\newtheorem{lemma}[theorem]{Lemma}
	\newtheorem{corollary}[theorem]{Corollary}
	\newtheorem{problem}[theorem]{Problem}
	\newtheorem{fact}[theorem]{Fact}
	\newtheorem{question}[theorem]{Question}
  \newtheorem{example}[theorem]{Example}
  \newtheorem{definition}[theorem]{Definition}
  \newtheorem{remark}[theorem]{Remark}


%%% Definition style: % normal text
	
	% Follow different counters

	% Exercise
	\newtheorem{exerciseEnv}{Exercise}[]
		\newenvironment{exercise}[1][] 			{\expandafter\exerciseEnv\if\relax\detokenize{#1}\relax\else[#1]\fi \addcontentsline{toc}{subsubsection}
	    		{\footnotesize \hspace{1em} 
					{\theexerciseEnv}
					{\if\relax\detokenize{#1}\relax\else #1\fi}}%
	        }
		{\endexerciseEnv}

	% Solution
	\newtheorem*{solutionEnv}{Solution}
		\ifdefined\showsolution
			\newenvironment{solution}
				{\pushQED{\qed}\solutionEnv}
				{\popQED\endsolutionEnv}
			\newenvironment{solution*}{\solutionEnv}{\endsolutionEnv}
		\else  % else hide it using \excludecomment from package comment
			\newenvironment{solution}{}{}
			\excludecomment{solution}
			\newenvironment{solution*}{}{}
			\excludecomment{solution*}
		\fi
		

	% Comment
	\let\comment\relax  % so \comment{} doesn't clash with package comment
	\ifdefined\showcomment % if \showcomment, print environment
		\newtheorem*{comment}{\normalfont\emph{Comment}}
	\else  % else hide it using \excludecomment from package comment
		\newenvironment{comment}{}{}
		\excludecomment{comment}
	\fi
